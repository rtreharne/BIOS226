% Options for packages loaded elsewhere
\PassOptionsToPackage{unicode}{hyperref}
\PassOptionsToPackage{hyphens}{url}
%
\documentclass[
]{article}
\usepackage{amsmath,amssymb}
\usepackage{iftex}
\ifPDFTeX
  \usepackage[T1]{fontenc}
  \usepackage[utf8]{inputenc}
  \usepackage{textcomp} % provide euro and other symbols
\else % if luatex or xetex
  \usepackage{unicode-math} % this also loads fontspec
  \defaultfontfeatures{Scale=MatchLowercase}
  \defaultfontfeatures[\rmfamily]{Ligatures=TeX,Scale=1}
\fi
\usepackage{lmodern}
\ifPDFTeX\else
  % xetex/luatex font selection
\fi
% Use upquote if available, for straight quotes in verbatim environments
\IfFileExists{upquote.sty}{\usepackage{upquote}}{}
\IfFileExists{microtype.sty}{% use microtype if available
  \usepackage[]{microtype}
  \UseMicrotypeSet[protrusion]{basicmath} % disable protrusion for tt fonts
}{}
\makeatletter
\@ifundefined{KOMAClassName}{% if non-KOMA class
  \IfFileExists{parskip.sty}{%
    \usepackage{parskip}
  }{% else
    \setlength{\parindent}{0pt}
    \setlength{\parskip}{6pt plus 2pt minus 1pt}}
}{% if KOMA class
  \KOMAoptions{parskip=half}}
\makeatother
\usepackage{xcolor}
\usepackage[margin=1in]{geometry}
\setlength{\emergencystretch}{3em} % prevent overfull lines
\providecommand{\tightlist}{%
  \setlength{\itemsep}{0pt}\setlength{\parskip}{0pt}}
\setcounter{secnumdepth}{-\maxdimen} % remove section numbering
\ifLuaTeX
  \usepackage{selnolig}  % disable illegal ligatures
\fi
\IfFileExists{bookmark.sty}{\usepackage{bookmark}}{\usepackage{hyperref}}
\IfFileExists{xurl.sty}{\usepackage{xurl}}{} % add URL line breaks if available
\urlstyle{same}
\hypersetup{
  hidelinks,
  pdfcreator={LaTeX via pandoc}}

\author{}
\date{}

\begin{document}

\section{BIOS226 - Topic 5 - Supervised Learning (Part 1)}

\subsection{Student Companion Guide}

\subsubsection{Foundations for Biological Model Evaluation}

This companion follows the Part 1 lecture titles and explains each idea
in plain language. Use it alongside the slides to build confidence
before the Part 2 pipeline session.

\begin{center}\rule{0.5\linewidth}{0.5pt}\end{center}

\subsection{From Exploration to Prediction}

In exploratory analysis, we ask what structure is in the data (for
example PCA and clustering). In supervised learning, we ask a different
question: can we predict an outcome for a new patient from measured
features?

\begin{itemize}
\tightlist
\item
  Exploration: pattern finding without outcome labels.
\item
  Prediction: learning from labelled examples to support decisions.
\end{itemize}

In breast cancer, this move from description to prediction has had
direct treatment impact (Paik et al., 2004; Sparano et al., 2018).

\subsection{Aside: Oncotype DX in Plain Language}

Oncotype DX is a lab test that measures expression of a small group of
genes in a breast tumour sample.

\begin{itemize}
\tightlist
\item
  It gives a recurrence score to estimate how likely the cancer is to
  return.
\item
  Clinicians use that score, with other clinical information, to discuss
  whether chemotherapy is likely to help.
\item
  For many patients, this supports avoiding chemotherapy when expected
  benefit is low.
\end{itemize}

The evidence base developed in stages: early validation work (Paik et
al., 2004), then large prospective studies refining who benefits from
chemotherapy (Sparano et al., 2018; Kalinsky et al., 2021).

At global scale, the test has been used in more than 2 million patients
across over 100 countries over about 22 years, according to company
reporting, with an estimated 1.6 million people avoiding potentially
unnecessary chemotherapy (Exact Sciences, 2026).

\begin{center}\rule{0.5\linewidth}{0.5pt}\end{center}

\subsection{Unsupervised vs Supervised}

Unsupervised methods group samples by similarity. Supervised methods
learn relationships between features and known outcomes.

This distinction matters in cancer research: molecular classes can
change how disease is defined and treated (Perou et al., 2000; Hoadley
et al., 2018).

\begin{center}\rule{0.5\linewidth}{0.5pt}\end{center}

\subsection{What Are X and Y?}

Supervised learning uses:

\begin{itemize}
\tightlist
\item
  \(X\): input features (for example SNPs, gene expression, clinical
  variables).
\item
  \(Y\): target outcome (for example disease class, risk, or response).
\end{itemize}

Polygenic risk scores are a good example: many small SNP effects can add
up to meaningful risk differences (Khera et al., 2018).

\begin{center}\rule{0.5\linewidth}{0.5pt}\end{center}

\subsection{Classification vs Regression}

\begin{itemize}
\tightlist
\item
  Classification predicts categories (for example subtype A vs subtype
  B).
\item
  Regression predicts numeric values (for example a response score).
\end{itemize}

In practice, teams often model probabilities and then choose thresholds
for clinical decision-making.

\begin{center}\rule{0.5\linewidth}{0.5pt}\end{center}

\subsection{What Is a Model?}

A model is a mathematical mapping from \(X\) to \(Y\). During training,
it learns patterns from observed data. During validation, we test whether
those patterns generalise to unseen data.

The key point for biology: a useful model is not just accurate on one
dataset. It must also be robust across cohorts and measurement settings.

\begin{center}\rule{0.5\linewidth}{0.5pt}\end{center}

\subsection{The High-Dimensional Problem (p >> n)}

In many biological studies, features (\(p\)) greatly outnumber samples
(\(n\)). This increases the risk of overfitting.

\begin{itemize}
\tightlist
\item
  The model can learn noise instead of true biology.
\item
  Internal performance may look strong but fail in external data.
\item
  Feature signatures may not replicate.
\end{itemize}

That is why careful validation and leakage control are central to good
practice.

\begin{center}\rule{0.5\linewidth}{0.5pt}\end{center}

\subsection{Why Biology Is Harder}

Biological datasets often include technical variation, batch effects,
small sample sizes, and real biological heterogeneity.

So model quality depends on study design and evaluation discipline, not
just algorithm choice.

\begin{center}\rule{0.5\linewidth}{0.5pt}\end{center}

\subsection{Key Take-Home Messages}

\begin{itemize}
\tightlist
\item
  Supervised learning supports decisions, not just predictions.
\item
  Always separate training logic from final testing logic.
\item
  In biology, validation design is as important as model selection.
\item
  Oncotype DX is a concrete example of supervised learning affecting care
  at global scale.
\end{itemize}

\begin{center}\rule{0.5\linewidth}{0.5pt}\end{center}

\subsection{References}

\begin{itemize}
\tightlist
\item
  Hoadley, K.A., et al. (2018) 'Cell-of-origin patterns dominate the
  molecular classification of 10,000 tumours from 33 types of cancer',
  \emph{Cell}, 173(2), pp. 291--304.e6.
\item
  Kalinsky, K., et al. (2021) '21-Gene assay to inform chemotherapy
  benefit in node-positive breast cancer', \emph{New England Journal of
  Medicine}, 385(25), pp. 2336--2347.
\item
  Khera, A.V., et al. (2018) 'Genome-wide polygenic scores for common
  diseases identify individuals with risk equivalent to monogenic
  mutations', \emph{Nature Genetics}, 50(9), pp. 1219--1224.
\item
  Paik, S., et al. (2004) 'A multigene assay to predict recurrence of
  tamoxifen-treated, node-negative breast cancer', \emph{New England
  Journal of Medicine}, 351(27), pp. 2817--2826.
\item
  Perou, C.M., et al. (2000) 'Molecular portraits of human breast
  tumours', \emph{Nature}, 406(6797), pp. 747--752.
\item
  Sparano, J.A., et al. (2018) 'Adjuvant chemotherapy guided by a
  21-gene expression assay in breast cancer', \emph{New England Journal
  of Medicine}, 379(2), pp. 111--121.
\item
  Exact Sciences (2026) \emph{Oncotype DX Breast Recurrence Score Test
  Surpasses 2 Million Patients Worldwide}. Available at:
  \url{https://investor.exactsciences.com/investor-relations/press-releases/press-release-details/2026/Oncotype-DX-Breast-Recurrence-Score-Test-Surpasses-2-Million-Patients-Worldwide/default.aspx}
  (Accessed 22 February 2026).
\end{itemize}

\end{document}
