% Options for packages loaded elsewhere
\PassOptionsToPackage{unicode}{hyperref}
\PassOptionsToPackage{hyphens}{url}
%
\documentclass[
]{article}
\usepackage{amsmath,amssymb}
\usepackage{iftex}
\ifPDFTeX
  \usepackage[T1]{fontenc}
  \usepackage[utf8]{inputenc}
  \usepackage{textcomp} % provide euro and other symbols
\else % if luatex or xetex
  \usepackage{unicode-math} % this also loads fontspec
  \defaultfontfeatures{Scale=MatchLowercase}
  \defaultfontfeatures[\rmfamily]{Ligatures=TeX,Scale=1}
\fi
\usepackage{lmodern}
\ifPDFTeX\else
  % xetex/luatex font selection
\fi
% Use upquote if available, for straight quotes in verbatim environments
\IfFileExists{upquote.sty}{\usepackage{upquote}}{}
\IfFileExists{microtype.sty}{% use microtype if available
  \usepackage[]{microtype}
  \UseMicrotypeSet[protrusion]{basicmath} % disable protrusion for tt fonts
}{}
\makeatletter
\@ifundefined{KOMAClassName}{% if non-KOMA class
  \IfFileExists{parskip.sty}{%
    \usepackage{parskip}
  }{% else
    \setlength{\parindent}{0pt}
    \setlength{\parskip}{6pt plus 2pt minus 1pt}}
}{% if KOMA class
  \KOMAoptions{parskip=half}}
\makeatother
\usepackage{xcolor}
\usepackage[margin=1in]{geometry}
\usepackage{color}
\usepackage{fancyvrb}
\newcommand{\VerbBar}{|}
\newcommand{\VERB}{\Verb[commandchars=\\\{\}]}
\DefineVerbatimEnvironment{Highlighting}{Verbatim}{commandchars=\\\{\}}
% Add ',fontsize=\small' for more characters per line
\usepackage{framed}
\definecolor{shadecolor}{RGB}{248,248,248}
\newenvironment{Shaded}{\begin{snugshade}}{\end{snugshade}}
\newcommand{\AlertTok}[1]{\textcolor[rgb]{0.94,0.16,0.16}{#1}}
\newcommand{\AnnotationTok}[1]{\textcolor[rgb]{0.56,0.35,0.01}{\textbf{\textit{#1}}}}
\newcommand{\AttributeTok}[1]{\textcolor[rgb]{0.13,0.29,0.53}{#1}}
\newcommand{\BaseNTok}[1]{\textcolor[rgb]{0.00,0.00,0.81}{#1}}
\newcommand{\BuiltInTok}[1]{#1}
\newcommand{\CharTok}[1]{\textcolor[rgb]{0.31,0.60,0.02}{#1}}
\newcommand{\CommentTok}[1]{\textcolor[rgb]{0.56,0.35,0.01}{\textit{#1}}}
\newcommand{\CommentVarTok}[1]{\textcolor[rgb]{0.56,0.35,0.01}{\textbf{\textit{#1}}}}
\newcommand{\ConstantTok}[1]{\textcolor[rgb]{0.56,0.35,0.01}{#1}}
\newcommand{\ControlFlowTok}[1]{\textcolor[rgb]{0.13,0.29,0.53}{\textbf{#1}}}
\newcommand{\DataTypeTok}[1]{\textcolor[rgb]{0.13,0.29,0.53}{#1}}
\newcommand{\DecValTok}[1]{\textcolor[rgb]{0.00,0.00,0.81}{#1}}
\newcommand{\DocumentationTok}[1]{\textcolor[rgb]{0.56,0.35,0.01}{\textbf{\textit{#1}}}}
\newcommand{\ErrorTok}[1]{\textcolor[rgb]{0.64,0.00,0.00}{\textbf{#1}}}
\newcommand{\ExtensionTok}[1]{#1}
\newcommand{\FloatTok}[1]{\textcolor[rgb]{0.00,0.00,0.81}{#1}}
\newcommand{\FunctionTok}[1]{\textcolor[rgb]{0.13,0.29,0.53}{\textbf{#1}}}
\newcommand{\ImportTok}[1]{#1}
\newcommand{\InformationTok}[1]{\textcolor[rgb]{0.56,0.35,0.01}{\textbf{\textit{#1}}}}
\newcommand{\KeywordTok}[1]{\textcolor[rgb]{0.13,0.29,0.53}{\textbf{#1}}}
\newcommand{\NormalTok}[1]{#1}
\newcommand{\OperatorTok}[1]{\textcolor[rgb]{0.81,0.36,0.00}{\textbf{#1}}}
\newcommand{\OtherTok}[1]{\textcolor[rgb]{0.56,0.35,0.01}{#1}}
\newcommand{\PreprocessorTok}[1]{\textcolor[rgb]{0.56,0.35,0.01}{\textit{#1}}}
\newcommand{\RegionMarkerTok}[1]{#1}
\newcommand{\SpecialCharTok}[1]{\textcolor[rgb]{0.81,0.36,0.00}{\textbf{#1}}}
\newcommand{\SpecialStringTok}[1]{\textcolor[rgb]{0.31,0.60,0.02}{#1}}
\newcommand{\StringTok}[1]{\textcolor[rgb]{0.31,0.60,0.02}{#1}}
\newcommand{\VariableTok}[1]{\textcolor[rgb]{0.00,0.00,0.00}{#1}}
\newcommand{\VerbatimStringTok}[1]{\textcolor[rgb]{0.31,0.60,0.02}{#1}}
\newcommand{\WarningTok}[1]{\textcolor[rgb]{0.56,0.35,0.01}{\textbf{\textit{#1}}}}
\usepackage{graphicx}
\makeatletter
\def\maxwidth{\ifdim\Gin@nat@width>\linewidth\linewidth\else\Gin@nat@width\fi}
\def\maxheight{\ifdim\Gin@nat@height>\textheight\textheight\else\Gin@nat@height\fi}
\makeatother
% Scale images if necessary, so that they will not overflow the page
% margins by default, and it is still possible to overwrite the defaults
% using explicit options in \includegraphics[width, height, ...]{}
\setkeys{Gin}{width=\maxwidth,height=\maxheight,keepaspectratio}
% Set default figure placement to htbp
\makeatletter
\def\fps@figure{htbp}
\makeatother
\setlength{\emergencystretch}{3em} % prevent overfull lines
\providecommand{\tightlist}{%
  \setlength{\itemsep}{0pt}\setlength{\parskip}{0pt}}
\setcounter{secnumdepth}{-\maxdimen} % remove section numbering
\ifLuaTeX
  \usepackage{selnolig}  % disable illegal ligatures
\fi
\IfFileExists{bookmark.sty}{\usepackage{bookmark}}{\usepackage{hyperref}}
\IfFileExists{xurl.sty}{\usepackage{xurl}}{} % add URL line breaks if available
\urlstyle{same}
\hypersetup{
  pdftitle={BIOS226 Topic 5 --- Supervised Learning Workshop},
  hidelinks,
  pdfcreator={LaTeX via pandoc}}

\title{BIOS226 Topic 5 --- Supervised Learning Workshop}
\author{}
\date{\vspace{-2.5em}}

\begin{document}
\maketitle

{
\setcounter{tocdepth}{3}
\tableofcontents
}
\hypertarget{workshop-overview-2-hours}{%
\section{Workshop overview (2 hours)}\label{workshop-overview-2-hours}}

In this workshop you will run a complete, \textbf{leakage-safe}
supervised learning workflow on a \textbf{synthetic gene expression}
dataset:

\begin{enumerate}
\def\labelenumi{\arabic{enumi}.}
\tightlist
\item
  Generate synthetic tumor subtype data (high-dimensional: \emph{p
  \textgreater\textgreater{} n}).
\item
  Train and evaluate a classifier \textbf{without leaking test
  information into training}.
\item
  Interpret confusion matrices and ROC/AUC, and see how
  \textbf{threshold choice} changes clinical tradeoffs.
\item
  Reproduce common failure modes from the ``How to Fail'' lecture using
  the provided scenario script.
\end{enumerate}

\hypertarget{r-notebook-quick-primer-2-minutes}{%
\subsection{R Notebook quick primer (2
minutes)}\label{r-notebook-quick-primer-2-minutes}}

\begin{itemize}
\tightlist
\item
  What it is: An R Markdown document that mixes prose, code chunks, and
  outputs; when you knit or run chunks, code executes and results appear
  inline. It is lighter-weight than a full Quarto book but more
  interactive than a static script.
\item
  How to run a chunk: Click the green play button to the right of a code
  chunk (or use Ctrl+Shift+Enter / Cmd+Shift+Enter to run the current
  chunk in RStudio). Output appears directly beneath the chunk.
\item
  Workspace: All chunks share the same R session, so objects created in
  one chunk (e.g., \texttt{df}, \texttt{results}) are available in later
  chunks. Re-running from the top is the best way to reset state.
\item
  Working directory: The setup chunk sets \texttt{root.dir} to the
  notebook folder. We create \texttt{workshop\_outputs/} there for
  generated CSVs and plots.
\item
  Why useful for this workshop: You can read an explanation, tweak
  parameters, re-run a chunk, and immediately see how metrics/plots
  change---ideal for exploring leakage, thresholds, and failure modes.
\end{itemize}

\hypertarget{learning-objectives-practical}{%
\subsection{Learning objectives
(practical)}\label{learning-objectives-practical}}

By the end, you will be able to:

\begin{itemize}
\tightlist
\item
  Generate a reproducible synthetic dataset and sanity-check it.
\item
  Run a leakage-safe pipeline: validate → stratified split → CV (train
  only) → feature selection (train only) → scaling (train only) → final
  test evaluation.
\item
  Read a confusion matrix and compute precision/sensitivity/specificity.
\item
  Interpret ROC curves and AUC, and explain what they do (and do not)
  guarantee.
\item
  Identify ``silent failure'' patterns: overfitting, underfitting, wrong
  labels, feature-selection leakage, and class imbalance.
\end{itemize}

\hypertarget{generate-the-synthetic-dataset}{%
\section{1) Generate the synthetic
dataset}\label{generate-the-synthetic-dataset}}

We will generate a synthetic dataset that behaves like
already-normalized (log-expression-like) values:

\begin{itemize}
\tightlist
\item
  \texttt{n\ =\ 120} patients
\item
  \texttt{p\ =\ 10,000} genes
\item
  two subtypes: \texttt{Luminal\_A} vs \texttt{Basal\_like}
\item
  only \texttt{50} genes truly carry signal; the rest are noise
\end{itemize}

\begin{Shaded}
\begin{Highlighting}[]
\NormalTok{seed }\OtherTok{\textless{}{-}} \DecValTok{123}
\NormalTok{n\_samples }\OtherTok{\textless{}{-}} \DecValTok{120}
\NormalTok{n\_genes }\OtherTok{\textless{}{-}} \DecValTok{10000}
\NormalTok{n\_informative }\OtherTok{\textless{}{-}} \DecValTok{50}
\NormalTok{class\_proportion }\OtherTok{\textless{}{-}} \FloatTok{0.6}
\NormalTok{noise\_sd }\OtherTok{\textless{}{-}} \FloatTok{1.5}

\NormalTok{old\_wd }\OtherTok{\textless{}{-}} \FunctionTok{getwd}\NormalTok{()}
\FunctionTok{setwd}\NormalTok{(out\_dir)}
\FunctionTok{on.exit}\NormalTok{(}\FunctionTok{setwd}\NormalTok{(old\_wd), }\AttributeTok{add =} \ConstantTok{TRUE}\NormalTok{)}
\NormalTok{df }\OtherTok{\textless{}{-}} \FunctionTok{generate\_tumor\_dataset}\NormalTok{(}
  \AttributeTok{seed =}\NormalTok{ seed,}
  \AttributeTok{n\_samples =}\NormalTok{ n\_samples,}
  \AttributeTok{n\_genes =}\NormalTok{ n\_genes,}
  \AttributeTok{n\_informative =}\NormalTok{ n\_informative,}
  \AttributeTok{class\_proportion =}\NormalTok{ class\_proportion,}
  \AttributeTok{noise\_sd =}\NormalTok{ noise\_sd}
\NormalTok{)}

\FunctionTok{dim}\NormalTok{(df)}
\end{Highlighting}
\end{Shaded}

\begin{verbatim}
## [1]   120 10002
\end{verbatim}

\begin{Shaded}
\begin{Highlighting}[]
\FunctionTok{table}\NormalTok{(df}\SpecialCharTok{$}\NormalTok{Subtype)}
\end{Highlighting}
\end{Shaded}

\begin{verbatim}
## 
##  Luminal_A Basal_like 
##         72         48
\end{verbatim}

\begin{Shaded}
\begin{Highlighting}[]
\NormalTok{gene\_cols }\OtherTok{\textless{}{-}} \FunctionTok{grep}\NormalTok{(}\StringTok{"\^{}Gene\_"}\NormalTok{, }\FunctionTok{names}\NormalTok{(df), }\AttributeTok{value =} \ConstantTok{TRUE}\NormalTok{)}
\FunctionTok{length}\NormalTok{(gene\_cols)}
\end{Highlighting}
\end{Shaded}

\begin{verbatim}
## [1] 10000
\end{verbatim}

Exercise 1 (predict first): make it harder/easier

Change \texttt{noise\_sd} and/or \texttt{n\_informative} and predict
what will happen to AUC.

Hints:

\begin{itemize}
\tightlist
\item
  Increasing \texttt{noise\_sd} makes classes overlap more (harder
  task).
\item
  Decreasing \texttt{n\_informative} reduces true signal (harder task).
\end{itemize}

\begin{Shaded}
\begin{Highlighting}[]
\CommentTok{\# Example: harder dataset}
\NormalTok{noise\_sd }\OtherTok{\textless{}{-}} \FloatTok{2.5}
\NormalTok{n\_informative }\OtherTok{\textless{}{-}} \DecValTok{10}

\NormalTok{old\_wd }\OtherTok{\textless{}{-}} \FunctionTok{getwd}\NormalTok{()}
\FunctionTok{setwd}\NormalTok{(out\_dir)}
\FunctionTok{on.exit}\NormalTok{(}\FunctionTok{setwd}\NormalTok{(old\_wd), }\AttributeTok{add =} \ConstantTok{TRUE}\NormalTok{)}
\NormalTok{df\_harder }\OtherTok{\textless{}{-}} \FunctionTok{generate\_tumor\_dataset}\NormalTok{(}
  \AttributeTok{seed =}\NormalTok{ seed,}
  \AttributeTok{n\_samples =}\NormalTok{ n\_samples,}
  \AttributeTok{n\_genes =}\NormalTok{ n\_genes,}
  \AttributeTok{n\_informative =}\NormalTok{ n\_informative,}
  \AttributeTok{class\_proportion =}\NormalTok{ class\_proportion,}
  \AttributeTok{noise\_sd =}\NormalTok{ noise\_sd}
\NormalTok{)}
\end{Highlighting}
\end{Shaded}

\hypertarget{run-the-leakage-safe-pipeline-end-to-end}{%
\section{2) Run the leakage-safe pipeline
end-to-end}\label{run-the-leakage-safe-pipeline-end-to-end}}

We will treat \texttt{Basal\_like} as the \textbf{positive class}.

Key discipline:

\begin{itemize}
\tightlist
\item
  The \textbf{test set is held out} and not used for feature selection,
  scaling, or model tuning.
\item
  Cross-validation is performed \textbf{only inside the training split}.
\end{itemize}

\begin{Shaded}
\begin{Highlighting}[]
\NormalTok{results }\OtherTok{\textless{}{-}} \FunctionTok{run\_tumor\_workflow}\NormalTok{(}
  \AttributeTok{df =}\NormalTok{ df,}
  \AttributeTok{seed =} \DecValTok{123}\NormalTok{,}
  \AttributeTok{train\_fraction =} \FloatTok{0.8}\NormalTok{,}
  \AttributeTok{k\_folds =} \DecValTok{5}\NormalTok{,}
  \AttributeTok{top\_k\_genes =} \DecValTok{25}\NormalTok{,}
  \AttributeTok{positive\_class =} \StringTok{"Basal\_like"}\NormalTok{,}
  \AttributeTok{threshold =} \FloatTok{0.85}
\NormalTok{)}

\NormalTok{results}\SpecialCharTok{$}\NormalTok{dataset\_summary}
\end{Highlighting}
\end{Shaded}

\begin{verbatim}
## $n_samples
## [1] 120
## 
## $n_genes
## [1] 10000
## 
## $class_counts
## 
##  Luminal_A Basal_like 
##         72         48
\end{verbatim}

\begin{Shaded}
\begin{Highlighting}[]
\NormalTok{results}\SpecialCharTok{$}\NormalTok{split\_summary}
\end{Highlighting}
\end{Shaded}

\begin{verbatim}
## $train_n
## [1] 95
## 
## $test_n
## [1] 25
## 
## $train_class_counts
## labels_train
## Basal_like  Luminal_A 
##         38         57 
## 
## $test_class_counts
## labels_test
## Basal_like  Luminal_A 
##         10         15 
## 
## $train_idx
##  [1]   1   2   3   5   6   7   8   9  10  11  12  13  14  15  16  17  19  21  22
## [20]  23  24  26  27  29  30  32  33  35  36  37  38  39  40  42  43  44  45  47
## [39]  49  50  51  52  53  54  55  56  57  58  59  60  61  62  64  65  66  67  68
## [58]  69  70  71  73  74  75  76  81  83  85  88  89  90  91  93  94  95  96  98
## [77]  99 101 102 103 104 105 107 108 109 110 111 112 113 114 115 116 117 118 119
## 
## $test_idx
##  [1]   4  18  20  25  28  31  34  41  46  48  63  72  77  78  79  80  82  84  86
## [20]  87  92  97 100 106 120
\end{verbatim}

\begin{Shaded}
\begin{Highlighting}[]
\NormalTok{results}\SpecialCharTok{$}\NormalTok{cv}\SpecialCharTok{$}\NormalTok{auc\_values}
\end{Highlighting}
\end{Shaded}

\begin{verbatim}
## [1] 0.8489583 0.9947917 0.9147727 1.0000000 0.8896104
\end{verbatim}

\begin{Shaded}
\begin{Highlighting}[]
\NormalTok{results}\SpecialCharTok{$}\NormalTok{cv}\SpecialCharTok{$}\NormalTok{precision\_values}
\end{Highlighting}
\end{Shaded}

\begin{verbatim}
## [1] 0.8000000 0.8750000 0.6000000 1.0000000 0.7142857
\end{verbatim}

\begin{Shaded}
\begin{Highlighting}[]
\FunctionTok{c}\NormalTok{(}\AttributeTok{auc\_mean =}\NormalTok{ results}\SpecialCharTok{$}\NormalTok{cv}\SpecialCharTok{$}\NormalTok{auc\_mean, }\AttributeTok{auc\_sd =}\NormalTok{ results}\SpecialCharTok{$}\NormalTok{cv}\SpecialCharTok{$}\NormalTok{auc\_sd)}
\end{Highlighting}
\end{Shaded}

\begin{verbatim}
##   auc_mean     auc_sd 
## 0.92962662 0.06619706
\end{verbatim}

\begin{Shaded}
\begin{Highlighting}[]
\FunctionTok{c}\NormalTok{(}\AttributeTok{precision\_mean =}\NormalTok{ results}\SpecialCharTok{$}\NormalTok{cv}\SpecialCharTok{$}\NormalTok{precision\_mean, }\AttributeTok{precision\_sd =}\NormalTok{ results}\SpecialCharTok{$}\NormalTok{cv}\SpecialCharTok{$}\NormalTok{precision\_sd)}
\end{Highlighting}
\end{Shaded}

\begin{verbatim}
## precision_mean   precision_sd 
##      0.7978571      0.1524377
\end{verbatim}

\begin{Shaded}
\begin{Highlighting}[]
\NormalTok{results}\SpecialCharTok{$}\NormalTok{selected\_genes[}\DecValTok{1}\SpecialCharTok{:}\DecValTok{10}\NormalTok{]}
\end{Highlighting}
\end{Shaded}

\begin{verbatim}
##  [1] "Gene_33" "Gene_12" "Gene_22" "Gene_5"  "Gene_45" "Gene_4"  "Gene_16"
##  [8] "Gene_26" "Gene_50" "Gene_48"
\end{verbatim}

\hypertarget{confusion-matrix-threshold-tradeoffs}{%
\section{3) Confusion matrix + threshold
tradeoffs}\label{confusion-matrix-threshold-tradeoffs}}

The confusion matrix depends on a chosen probability threshold (here,
\texttt{0.85}).

\begin{Shaded}
\begin{Highlighting}[]
\NormalTok{results}\SpecialCharTok{$}\NormalTok{test}\SpecialCharTok{$}\NormalTok{confusion\_matrix}
\end{Highlighting}
\end{Shaded}

\begin{verbatim}
##             Actual
## Predicted    Luminal_A Basal_like
##   Luminal_A         12          1
##   Basal_like         3          9
\end{verbatim}

\begin{Shaded}
\begin{Highlighting}[]
\NormalTok{confusion\_metrics }\OtherTok{\textless{}{-}} \ControlFlowTok{function}\NormalTok{(confusion\_matrix\_2x2) \{}
\NormalTok{  mat }\OtherTok{\textless{}{-}} \FunctionTok{as.matrix}\NormalTok{(confusion\_matrix\_2x2)}
  \ControlFlowTok{if}\NormalTok{ (}\SpecialCharTok{!}\FunctionTok{all}\NormalTok{(}\FunctionTok{dim}\NormalTok{(mat) }\SpecialCharTok{==} \FunctionTok{c}\NormalTok{(}\DecValTok{2}\NormalTok{, }\DecValTok{2}\NormalTok{))) }\FunctionTok{stop}\NormalTok{(}\StringTok{"Expected a 2x2 confusion matrix."}\NormalTok{)}

\NormalTok{  tn }\OtherTok{\textless{}{-}}\NormalTok{ mat[}\DecValTok{1}\NormalTok{, }\DecValTok{1}\NormalTok{]}
\NormalTok{  fn }\OtherTok{\textless{}{-}}\NormalTok{ mat[}\DecValTok{1}\NormalTok{, }\DecValTok{2}\NormalTok{]}
\NormalTok{  fp }\OtherTok{\textless{}{-}}\NormalTok{ mat[}\DecValTok{2}\NormalTok{, }\DecValTok{1}\NormalTok{]}
\NormalTok{  tp }\OtherTok{\textless{}{-}}\NormalTok{ mat[}\DecValTok{2}\NormalTok{, }\DecValTok{2}\NormalTok{]}

\NormalTok{  safe\_ratio }\OtherTok{\textless{}{-}} \ControlFlowTok{function}\NormalTok{(num, den) }\ControlFlowTok{if}\NormalTok{ (den }\SpecialCharTok{==} \DecValTok{0}\NormalTok{) }\ConstantTok{NA\_real\_} \ControlFlowTok{else}\NormalTok{ num }\SpecialCharTok{/}\NormalTok{ den}

\NormalTok{  precision }\OtherTok{\textless{}{-}} \FunctionTok{safe\_ratio}\NormalTok{(tp, tp }\SpecialCharTok{+}\NormalTok{ fp)}
\NormalTok{  sensitivity }\OtherTok{\textless{}{-}} \FunctionTok{safe\_ratio}\NormalTok{(tp, tp }\SpecialCharTok{+}\NormalTok{ fn)}
\NormalTok{  specificity }\OtherTok{\textless{}{-}} \FunctionTok{safe\_ratio}\NormalTok{(tn, tn }\SpecialCharTok{+}\NormalTok{ fp)}

  \FunctionTok{list}\NormalTok{(}
    \AttributeTok{tn =}\NormalTok{ tn, }\AttributeTok{fn =}\NormalTok{ fn, }\AttributeTok{fp =}\NormalTok{ fp, }\AttributeTok{tp =}\NormalTok{ tp,}
    \AttributeTok{precision =}\NormalTok{ precision,}
    \AttributeTok{sensitivity =}\NormalTok{ sensitivity,}
    \AttributeTok{specificity =}\NormalTok{ specificity}
\NormalTok{  )}
\NormalTok{\}}

\NormalTok{cm }\OtherTok{\textless{}{-}} \FunctionTok{confusion\_metrics}\NormalTok{(results}\SpecialCharTok{$}\NormalTok{test}\SpecialCharTok{$}\NormalTok{confusion\_matrix)}
\NormalTok{cm}
\end{Highlighting}
\end{Shaded}

\begin{verbatim}
## $tn
## [1] 12
## 
## $fn
## [1] 1
## 
## $fp
## [1] 3
## 
## $tp
## [1] 9
## 
## $precision
## [1] 0.75
## 
## $sensitivity
## [1] 0.9
## 
## $specificity
## [1] 0.8
\end{verbatim}

\begin{Shaded}
\begin{Highlighting}[]
\NormalTok{truth\_table\_file }\OtherTok{\textless{}{-}} \FunctionTok{file.path}\NormalTok{(out\_dir, }\FunctionTok{paste0}\NormalTok{(}\StringTok{"truth\_table\_test\_seed\_"}\NormalTok{, results}\SpecialCharTok{$}\NormalTok{config}\SpecialCharTok{$}\NormalTok{seed, }\StringTok{".png"}\NormalTok{))}

\FunctionTok{save\_truth\_table\_plot}\NormalTok{(}
  \AttributeTok{confusion\_matrix =}\NormalTok{ results}\SpecialCharTok{$}\NormalTok{test}\SpecialCharTok{$}\NormalTok{confusion\_matrix,}
  \AttributeTok{file\_path =}\NormalTok{ truth\_table\_file}
\NormalTok{)}
\end{Highlighting}
\end{Shaded}

\begin{verbatim}
## pdf 
##   2
\end{verbatim}

\begin{Shaded}
\begin{Highlighting}[]
\NormalTok{knitr}\SpecialCharTok{::}\FunctionTok{include\_graphics}\NormalTok{(truth\_table\_file)}
\end{Highlighting}
\end{Shaded}

\includegraphics[width=12.5in]{workshop_outputs/truth_table_test_seed_123}

Exercise 2: change the threshold

Change \texttt{threshold} from \texttt{0.85} to \texttt{0.5}.

Questions:

\begin{itemize}
\tightlist
\item
  Which error type increases (FP or FN), and why?
\item
  Which metric improves, and which metric worsens?
\end{itemize}

\begin{Shaded}
\begin{Highlighting}[]
\NormalTok{results\_thr\_05 }\OtherTok{\textless{}{-}} \FunctionTok{run\_tumor\_workflow}\NormalTok{(}
  \AttributeTok{df =}\NormalTok{ df,}
  \AttributeTok{seed =} \DecValTok{123}\NormalTok{,}
  \AttributeTok{train\_fraction =} \FloatTok{0.8}\NormalTok{,}
  \AttributeTok{k\_folds =} \DecValTok{5}\NormalTok{,}
  \AttributeTok{top\_k\_genes =} \DecValTok{25}\NormalTok{,}
  \AttributeTok{positive\_class =} \StringTok{"Basal\_like"}\NormalTok{,}
  \AttributeTok{threshold =} \FloatTok{0.5}
\NormalTok{)}

\NormalTok{results\_thr\_05}\SpecialCharTok{$}\NormalTok{test}\SpecialCharTok{$}\NormalTok{confusion\_matrix}
\FunctionTok{confusion\_metrics}\NormalTok{(results\_thr\_05}\SpecialCharTok{$}\NormalTok{test}\SpecialCharTok{$}\NormalTok{confusion\_matrix)}
\end{Highlighting}
\end{Shaded}

\hypertarget{roc-curve-auc-threshold-free-ranking}{%
\section{4) ROC curve + AUC (threshold-free
ranking)}\label{roc-curve-auc-threshold-free-ranking}}

ROC/AUC evaluates how well the model \textbf{ranks} positives above
negatives across all thresholds.

\begin{Shaded}
\begin{Highlighting}[]
\NormalTok{roc\_file }\OtherTok{\textless{}{-}} \FunctionTok{file.path}\NormalTok{(out\_dir, }\FunctionTok{paste0}\NormalTok{(}\StringTok{"roc\_curve\_test\_seed\_"}\NormalTok{, results}\SpecialCharTok{$}\NormalTok{config}\SpecialCharTok{$}\NormalTok{seed, }\StringTok{".png"}\NormalTok{))}
\NormalTok{roc\_file\_ggplot }\OtherTok{\textless{}{-}} \FunctionTok{file.path}\NormalTok{(out\_dir, }\FunctionTok{paste0}\NormalTok{(}\StringTok{"roc\_curve\_test\_seed\_"}\NormalTok{, results}\SpecialCharTok{$}\NormalTok{config}\SpecialCharTok{$}\NormalTok{seed, }\StringTok{"\_ggplot2.png"}\NormalTok{))}

\FunctionTok{save\_roc\_plot}\NormalTok{(}
  \AttributeTok{fpr =}\NormalTok{ results}\SpecialCharTok{$}\NormalTok{test}\SpecialCharTok{$}\NormalTok{roc\_fpr,}
  \AttributeTok{tpr =}\NormalTok{ results}\SpecialCharTok{$}\NormalTok{test}\SpecialCharTok{$}\NormalTok{roc\_tpr,}
  \AttributeTok{auc\_value =}\NormalTok{ results}\SpecialCharTok{$}\NormalTok{test}\SpecialCharTok{$}\NormalTok{auc,}
  \AttributeTok{file\_path =}\NormalTok{ roc\_file,}
  \AttributeTok{cv\_roc\_curves =}\NormalTok{ results}\SpecialCharTok{$}\NormalTok{cv}\SpecialCharTok{$}\NormalTok{roc\_curves}
\NormalTok{)}
\end{Highlighting}
\end{Shaded}

\begin{verbatim}
## pdf 
##   2
\end{verbatim}

\begin{Shaded}
\begin{Highlighting}[]
\FunctionTok{save\_roc\_plot\_ggplot}\NormalTok{(}
  \AttributeTok{fpr =}\NormalTok{ results}\SpecialCharTok{$}\NormalTok{test}\SpecialCharTok{$}\NormalTok{roc\_fpr,}
  \AttributeTok{tpr =}\NormalTok{ results}\SpecialCharTok{$}\NormalTok{test}\SpecialCharTok{$}\NormalTok{roc\_tpr,}
  \AttributeTok{auc\_value =}\NormalTok{ results}\SpecialCharTok{$}\NormalTok{test}\SpecialCharTok{$}\NormalTok{auc,}
  \AttributeTok{file\_path =}\NormalTok{ roc\_file\_ggplot,}
  \AttributeTok{cv\_roc\_curves =}\NormalTok{ results}\SpecialCharTok{$}\NormalTok{cv}\SpecialCharTok{$}\NormalTok{roc\_curves}
\NormalTok{)}

\NormalTok{knitr}\SpecialCharTok{::}\FunctionTok{include\_graphics}\NormalTok{(roc\_file)}
\end{Highlighting}
\end{Shaded}

\includegraphics[width=11.11in]{workshop_outputs/roc_curve_test_seed_123}

\hypertarget{threshold-sweep-mini-lab-operating-point-selection}{%
\section{5) Threshold sweep mini-lab (operating point
selection)}\label{threshold-sweep-mini-lab-operating-point-selection}}

Now that the workflow returns held-out test probabilities
(\texttt{results\$test\$probs}) and truth labels
(\texttt{results\$test\$y\_true\_binary}), we can explore tradeoffs
across thresholds.

\begin{Shaded}
\begin{Highlighting}[]
\NormalTok{probs }\OtherTok{\textless{}{-}}\NormalTok{ results}\SpecialCharTok{$}\NormalTok{test}\SpecialCharTok{$}\NormalTok{probs}
\NormalTok{y\_true }\OtherTok{\textless{}{-}}\NormalTok{ results}\SpecialCharTok{$}\NormalTok{test}\SpecialCharTok{$}\NormalTok{y\_true\_binary}

\NormalTok{threshold\_grid }\OtherTok{\textless{}{-}} \FunctionTok{seq}\NormalTok{(}\FloatTok{0.05}\NormalTok{, }\FloatTok{0.95}\NormalTok{, }\AttributeTok{by =} \FloatTok{0.05}\NormalTok{)}

\NormalTok{threshold\_summary }\OtherTok{\textless{}{-}} \FunctionTok{do.call}\NormalTok{(}
\NormalTok{  rbind,}
  \FunctionTok{lapply}\NormalTok{(threshold\_grid, }\ControlFlowTok{function}\NormalTok{(thr) \{}
\NormalTok{    pred }\OtherTok{\textless{}{-}} \FunctionTok{ifelse}\NormalTok{(probs }\SpecialCharTok{\textgreater{}=}\NormalTok{ thr, 1L, 0L)}

\NormalTok{    tp }\OtherTok{\textless{}{-}} \FunctionTok{sum}\NormalTok{(pred }\SpecialCharTok{==}\NormalTok{ 1L }\SpecialCharTok{\&}\NormalTok{ y\_true }\SpecialCharTok{==}\NormalTok{ 1L)}
\NormalTok{    fp }\OtherTok{\textless{}{-}} \FunctionTok{sum}\NormalTok{(pred }\SpecialCharTok{==}\NormalTok{ 1L }\SpecialCharTok{\&}\NormalTok{ y\_true }\SpecialCharTok{==}\NormalTok{ 0L)}
\NormalTok{    tn }\OtherTok{\textless{}{-}} \FunctionTok{sum}\NormalTok{(pred }\SpecialCharTok{==}\NormalTok{ 0L }\SpecialCharTok{\&}\NormalTok{ y\_true }\SpecialCharTok{==}\NormalTok{ 0L)}
\NormalTok{    fn }\OtherTok{\textless{}{-}} \FunctionTok{sum}\NormalTok{(pred }\SpecialCharTok{==}\NormalTok{ 0L }\SpecialCharTok{\&}\NormalTok{ y\_true }\SpecialCharTok{==}\NormalTok{ 1L)}

\NormalTok{    safe\_ratio }\OtherTok{\textless{}{-}} \ControlFlowTok{function}\NormalTok{(num, den) }\ControlFlowTok{if}\NormalTok{ (den }\SpecialCharTok{==} \DecValTok{0}\NormalTok{) }\ConstantTok{NA\_real\_} \ControlFlowTok{else}\NormalTok{ num }\SpecialCharTok{/}\NormalTok{ den}

    \FunctionTok{data.frame}\NormalTok{(}
      \AttributeTok{threshold =}\NormalTok{ thr,}
      \AttributeTok{precision =} \FunctionTok{safe\_ratio}\NormalTok{(tp, tp }\SpecialCharTok{+}\NormalTok{ fp),}
      \AttributeTok{sensitivity =} \FunctionTok{safe\_ratio}\NormalTok{(tp, tp }\SpecialCharTok{+}\NormalTok{ fn),}
      \AttributeTok{specificity =} \FunctionTok{safe\_ratio}\NormalTok{(tn, tn }\SpecialCharTok{+}\NormalTok{ fp),}
      \AttributeTok{tp =}\NormalTok{ tp, }\AttributeTok{fp =}\NormalTok{ fp, }\AttributeTok{tn =}\NormalTok{ tn, }\AttributeTok{fn =}\NormalTok{ fn}
\NormalTok{    )}
\NormalTok{  \})}
\NormalTok{)}

\FunctionTok{head}\NormalTok{(threshold\_summary, }\DecValTok{6}\NormalTok{)}
\end{Highlighting}
\end{Shaded}

\begin{verbatim}
##   threshold precision sensitivity specificity tp fp tn fn
## 1      0.05 0.6428571         0.9   0.6666667  9  5 10  1
## 2      0.10 0.6923077         0.9   0.7333333  9  4 11  1
## 3      0.15 0.6923077         0.9   0.7333333  9  4 11  1
## 4      0.20 0.6923077         0.9   0.7333333  9  4 11  1
## 5      0.25 0.6923077         0.9   0.7333333  9  4 11  1
## 6      0.30 0.6923077         0.9   0.7333333  9  4 11  1
\end{verbatim}

\begin{Shaded}
\begin{Highlighting}[]
\NormalTok{op }\OtherTok{\textless{}{-}} \FunctionTok{par}\NormalTok{(}\AttributeTok{mar =} \FunctionTok{c}\NormalTok{(}\FloatTok{4.5}\NormalTok{, }\FloatTok{4.5}\NormalTok{, }\DecValTok{2}\NormalTok{, }\DecValTok{1}\NormalTok{))}
\FunctionTok{plot}\NormalTok{(}
\NormalTok{  threshold\_summary}\SpecialCharTok{$}\NormalTok{threshold,}
\NormalTok{  threshold\_summary}\SpecialCharTok{$}\NormalTok{sensitivity,}
  \AttributeTok{type =} \StringTok{"b"}\NormalTok{,}
  \AttributeTok{pch =} \DecValTok{16}\NormalTok{,}
  \AttributeTok{ylim =} \FunctionTok{c}\NormalTok{(}\DecValTok{0}\NormalTok{, }\DecValTok{1}\NormalTok{),}
  \AttributeTok{xlab =} \StringTok{"Threshold"}\NormalTok{,}
  \AttributeTok{ylab =} \StringTok{"Metric value"}\NormalTok{,}
  \AttributeTok{main =} \StringTok{"Threshold tradeoffs (held{-}out test set)"}
\NormalTok{)}
\FunctionTok{lines}\NormalTok{(threshold\_summary}\SpecialCharTok{$}\NormalTok{threshold, threshold\_summary}\SpecialCharTok{$}\NormalTok{specificity, }\AttributeTok{type =} \StringTok{"b"}\NormalTok{, }\AttributeTok{pch =} \DecValTok{16}\NormalTok{, }\AttributeTok{col =} \StringTok{"\#1f5a99"}\NormalTok{)}
\FunctionTok{lines}\NormalTok{(threshold\_summary}\SpecialCharTok{$}\NormalTok{threshold, threshold\_summary}\SpecialCharTok{$}\NormalTok{precision, }\AttributeTok{type =} \StringTok{"b"}\NormalTok{, }\AttributeTok{pch =} \DecValTok{16}\NormalTok{, }\AttributeTok{col =} \StringTok{"gray30"}\NormalTok{)}
\FunctionTok{legend}\NormalTok{(}
  \StringTok{"bottomleft"}\NormalTok{,}
  \AttributeTok{legend =} \FunctionTok{c}\NormalTok{(}\StringTok{"Sensitivity"}\NormalTok{, }\StringTok{"Specificity"}\NormalTok{, }\StringTok{"Precision"}\NormalTok{),}
  \AttributeTok{col =} \FunctionTok{c}\NormalTok{(}\StringTok{"black"}\NormalTok{, }\StringTok{"\#1f5a99"}\NormalTok{, }\StringTok{"gray30"}\NormalTok{),}
  \AttributeTok{lty =} \DecValTok{1}\NormalTok{,}
  \AttributeTok{pch =} \DecValTok{16}\NormalTok{,}
  \AttributeTok{bty =} \StringTok{"n"}
\NormalTok{)}
\end{Highlighting}
\end{Shaded}

\includegraphics{BIOS226_Topic5_Supervised_Learning_Workshop_files/figure-latex/threshold-sweep-plot-1.pdf}

\begin{Shaded}
\begin{Highlighting}[]
\FunctionTok{par}\NormalTok{(op)}
\end{Highlighting}
\end{Shaded}

\textbf{Prompt:} Choose a threshold for each clinical goal:

\begin{itemize}
\tightlist
\item
  Goal A: minimize false positives (high specificity).
\item
  Goal B: minimize false negatives (high sensitivity).
\end{itemize}

\hypertarget{how-to-fail-run-the-full-scenario-script}{%
\section{6) ``How to Fail'' --- run the full scenario
script}\label{how-to-fail-run-the-full-scenario-script}}

This section reproduces the Part 3 ``How to Fail'' plots by running the
provided script.

\textbf{Note:} This chunk can take \textasciitilde1--3 minutes depending
on your machine.

\begin{Shaded}
\begin{Highlighting}[]
\ControlFlowTok{if}\NormalTok{ (}\SpecialCharTok{!}\FunctionTok{file.exists}\NormalTok{(scenario\_path)) }\FunctionTok{stop}\NormalTok{(}\FunctionTok{paste0}\NormalTok{(}\StringTok{"Missing script: "}\NormalTok{, scenario\_path))}
\FunctionTok{source}\NormalTok{(scenario\_path, }\AttributeTok{chdir =} \ConstantTok{TRUE}\NormalTok{)}
\end{Highlighting}
\end{Shaded}

\begin{verbatim}
## 
## === ROC Scenario Summary ===
##                     scenario auc_cv_mean  auc_test auc_train
##  0. Perfect (high n, high p)   0.9416184 0.9487847        NA
##               1. Overfitting   0.8917884 0.9333333         1
##              2. Underfitting   0.7765152 0.8533333        NA
##  3. Wrong labels in training   0.2251894 0.1600000        NA
##      4. Leakage before split   0.9612013 0.9966667        NA
##    5. Ignore 90:10 imbalance   0.8954248 0.4523810        NA
##                                                                                              notes
##                                          n=240; p=8000; informative=25; noise_sd=1.5; cv_auc=0.942
##                                                    top_k=50; cv_auc=0.892; train_vs_test_gap=0.067
##                                                  top_k=1; cv_auc=0.777; intentionally low capacity
##                                                            cv_auc=0.225; AUC below 0.5 as expected
##                                        leaked_auc=0.997; safe_auc=0.933; delta=0.063; cv_auc=0.961
##  train_counts={Basal_like:9, Luminal_A:87}; test_counts={Basal_like:3, Luminal_A:21}; cv_auc=0.895
##                                          output_file
##   roc_scenario_0_perfect_high_n_high_p_seed_1006.png
##             roc_scenario_1_overfitting_seed_1201.png
##            roc_scenario_2_underfitting_seed_1002.png
##            roc_scenario_3_wrong_labels_seed_1003.png
##         roc_scenario_4_feature_leakage_seed_1000.png
##  roc_scenario_5_ignore_imbalance_90_10_seed_1005.png
## 
## Generated ROC files:
##  - roc_scenario_0_perfect_high_n_high_p_seed_1006.png
##  - roc_scenario_1_overfitting_seed_1201.png
##  - roc_scenario_2_underfitting_seed_1002.png
##  - roc_scenario_3_wrong_labels_seed_1003.png
##  - roc_scenario_4_feature_leakage_seed_1000.png
##  - roc_scenario_5_ignore_imbalance_90_10_seed_1005.png
## Done.
\end{verbatim}

\begin{Shaded}
\begin{Highlighting}[]
\NormalTok{scenario\_files }\OtherTok{\textless{}{-}} \FunctionTok{c}\NormalTok{(}
  \FunctionTok{file.path}\NormalTok{(}\StringTok{"scripts"}\NormalTok{, }\StringTok{"roc\_scenario\_0\_perfect\_high\_n\_high\_p\_seed\_1006.png"}\NormalTok{),}
  \FunctionTok{file.path}\NormalTok{(}\StringTok{"scripts"}\NormalTok{, }\StringTok{"roc\_scenario\_1\_overfitting\_seed\_1201.png"}\NormalTok{),}
  \FunctionTok{file.path}\NormalTok{(}\StringTok{"scripts"}\NormalTok{, }\StringTok{"roc\_scenario\_2\_underfitting\_seed\_1002.png"}\NormalTok{),}
  \FunctionTok{file.path}\NormalTok{(}\StringTok{"scripts"}\NormalTok{, }\StringTok{"roc\_scenario\_3\_wrong\_labels\_seed\_1003.png"}\NormalTok{),}
  \FunctionTok{file.path}\NormalTok{(}\StringTok{"scripts"}\NormalTok{, }\StringTok{"roc\_scenario\_4\_feature\_leakage\_seed\_1000.png"}\NormalTok{),}
  \FunctionTok{file.path}\NormalTok{(}\StringTok{"scripts"}\NormalTok{, }\StringTok{"roc\_scenario\_5\_ignore\_imbalance\_90\_10\_seed\_1005.png"}\NormalTok{)}
\NormalTok{)}

\NormalTok{missing\_files }\OtherTok{\textless{}{-}}\NormalTok{ scenario\_files[}\SpecialCharTok{!}\FunctionTok{file.exists}\NormalTok{(scenario\_files)]}
\ControlFlowTok{if}\NormalTok{ (}\FunctionTok{length}\NormalTok{(missing\_files) }\SpecialCharTok{\textgreater{}} \DecValTok{0}\NormalTok{) \{}
  \FunctionTok{cat}\NormalTok{(}\StringTok{"Some scenario images were not found. Rerun the previous chunk.}\SpecialCharTok{\textbackslash{}n\textbackslash{}n}\StringTok{Missing:}\SpecialCharTok{\textbackslash{}n}\StringTok{"}\NormalTok{)}
  \FunctionTok{cat}\NormalTok{(}\FunctionTok{paste0}\NormalTok{(}\StringTok{"{-} "}\NormalTok{, missing\_files, }\AttributeTok{collapse =} \StringTok{"}\SpecialCharTok{\textbackslash{}n}\StringTok{"}\NormalTok{))}
\NormalTok{\} }\ControlFlowTok{else}\NormalTok{ \{}
\NormalTok{  knitr}\SpecialCharTok{::}\FunctionTok{include\_graphics}\NormalTok{(scenario\_files)}
\NormalTok{\}}
\end{Highlighting}
\end{Shaded}

\includegraphics[width=17.78in]{scripts/roc_scenario_0_perfect_high_n_high_p_seed_1006}
\includegraphics[width=17.78in]{scripts/roc_scenario_1_overfitting_seed_1201}
\includegraphics[width=17.78in]{scripts/roc_scenario_2_underfitting_seed_1002}
\includegraphics[width=17.78in]{scripts/roc_scenario_3_wrong_labels_seed_1003}
\includegraphics[width=17.78in]{scripts/roc_scenario_4_feature_leakage_seed_1000}
\includegraphics[width=17.78in]{scripts/roc_scenario_5_ignore_imbalance_90_10_seed_1005}

\hypertarget{interpretation-prompts-write-12-sentences-each}{%
\subsection{Interpretation prompts (write 1--2 sentences
each)}\label{interpretation-prompts-write-12-sentences-each}}

For each scenario:

\begin{enumerate}
\def\labelenumi{\arabic{enumi}.}
\tightlist
\item
  What do the \textbf{CV envelope} and \textbf{test ROC} suggest about
  generalisation?
\item
  Which failure mode is shown (overfitting / underfitting / wrong labels
  / leakage / imbalance)?
\item
  What pipeline decision caused the failure?
\end{enumerate}

\hypertarget{wrap-up-checklist}{%
\section{7) Wrap-up checklist}\label{wrap-up-checklist}}

You should now be able to answer:

\begin{itemize}
\tightlist
\item
  Where can leakage occur in this workflow (and how did we prevent it)?
\item
  Why can ROC/AUC look acceptable while the confusion matrix is
  clinically unacceptable?
\item
  What changed when we altered threshold, and why is this not
  ``cheating''?
\end{itemize}

\hypertarget{reproducibility}{%
\section{8) Reproducibility}\label{reproducibility}}

\begin{Shaded}
\begin{Highlighting}[]
\FunctionTok{sessionInfo}\NormalTok{()}
\end{Highlighting}
\end{Shaded}

\begin{verbatim}
## R version 4.3.3 (2024-02-29)
## Platform: x86_64-pc-linux-gnu (64-bit)
## Running under: Ubuntu 24.04.3 LTS
## 
## Matrix products: default
## BLAS:   /usr/lib/x86_64-linux-gnu/blas/libblas.so.3.12.0 
## LAPACK: /usr/lib/x86_64-linux-gnu/lapack/liblapack.so.3.12.0
## 
## locale:
##  [1] LC_CTYPE=en_US.UTF-8       LC_NUMERIC=C              
##  [3] LC_TIME=en_US.UTF-8        LC_COLLATE=en_US.UTF-8    
##  [5] LC_MONETARY=en_US.UTF-8    LC_MESSAGES=en_US.UTF-8   
##  [7] LC_PAPER=en_US.UTF-8       LC_NAME=C                 
##  [9] LC_ADDRESS=C               LC_TELEPHONE=C            
## [11] LC_MEASUREMENT=en_US.UTF-8 LC_IDENTIFICATION=C       
## 
## time zone: Europe/London
## tzcode source: system (glibc)
## 
## attached base packages:
## [1] stats     graphics  grDevices utils     datasets  methods   base     
## 
## loaded via a namespace (and not attached):
##  [1] vctrs_0.7.1        cli_3.6.5          knitr_1.51         rlang_1.1.7       
##  [5] xfun_0.56          otel_0.2.0         png_0.1-8          generics_0.1.4    
##  [9] S7_0.2.1           labeling_0.4.3     glue_1.8.0         htmltools_0.5.9   
## [13] scales_1.4.0       rmarkdown_2.30     grid_4.3.3         tibble_3.3.1      
## [17] evaluate_1.0.5     fastmap_1.2.0      yaml_2.3.12        lifecycle_1.0.5   
## [21] compiler_4.3.3     dplyr_1.1.4        RColorBrewer_1.1-3 pkgconfig_2.0.3   
## [25] farver_2.1.2       digest_0.6.39      R6_2.6.1           tidyselect_1.2.1  
## [29] pillar_1.11.1      magrittr_2.0.4     withr_3.0.2        tools_4.3.3       
## [33] gtable_0.3.6       ggplot2_4.0.1
\end{verbatim}

\end{document}
